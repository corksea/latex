\mfpicnumber{1}

\opengraphsfile{CartesianPlane}

\setcounter{footnote}{0}

\label{CartesianPlane}

\subsection{Sets of Numbers}
\label{SetsofNumbers}
While the authors would like nothing more than to delve quickly and deeply into the sheer excitement that is \textit{Precalculus}, experience\footnote{\ldots to be read  as `good, solid feedback from colleagues' \ldots} has taught us that a brief refresher on some basic notions is welcome, if not completely necessary, at this stage.  To that end, we present a brief summary of `set theory' and some of the associated vocabulary and notations we use in the text. Like all good Math books, we begin with a definition.

\colorbox{ResultColor}{\bbm

%\smallskip

\begin{defn} \label{setdef}

A \textbf{set}\index{set ! definition of} is a well-defined collection of objects which are called the `elements' of the set.  Here, `well-defined' means that it is possible to determine if something belongs to the collection or not, without prejudice. 

\end{defn}

\ebm}

\smallskip

For example, the collection of letters that make up the word ``smolko'' is well-defined and is a set, but  the collection of the worst math teachers in the world is \textbf{not} well-defined, and so is \textbf{not} a set.\footnote{For a more thought-provoking example, consider the collection of all things that do not contain themselves - this leads to the famous \href{http://en.wikipedia.org/wiki/Russell's_paradox}{\underline{Russell's Paradox}}.}  In general, there are three ways to describe sets.  They are

\smallskip

\colorbox{ResultColor}{\bbm

%\smallskip

\centerline{\textbf{Ways to Describe Sets}}

\begin{enumerate}

\item \textbf{The Verbal Method:} Use a sentence to define a set.\index{set ! verbal description}

\item \textbf{The Roster Method:}  Begin with a left brace `$\{$', list each element of the set \textit{only once} and then end with a right brace `$\}$'.\index{set ! roster method}

\item \textbf{The Set-Builder Method:} A combination of the verbal and roster methods using a ``dummy variable'' such as $x$.\index{set ! set-builder notation}\index{set-builder notation}

\end{enumerate}

\ebm}

\smallskip

For example, let $S$ be the set described \textit{verbally} as the set of letters that make up the word ``smolko''.  A  \textbf{roster} description of $S$ would be  $\left\{ s, m, o, l, k \right\}$. Note that we listed `o' only once, even though it appears twice in ``smolko.''  Also, the \textit{order} of the elements doesn't matter, so $\left\{ k,  l,  m, o, s \right\}$ is also a roster description of $S$.   A \textbf{set-builder} description of $S$ is: \[ \{ x \, | \, \mbox{$x$ is a
letter in the word ``smolko''.}\} \]

The way to read this is: `The set of elements $x$ \underline{such that} $x$ is a letter in the word ``smolko.'''   In each of the above cases, we may use the familiar equals sign `$=$' and write  $S = \left\{ s, m, o, l, k \right\}$ or $S = \{ x \, | \, \mbox{$x$ is a
letter in the word ``smolko''.}\}$.  Clearly $m$ is in $S$ and $q$ is not in $S$.  We express these sentiments mathematically by writing  $m \in S$ and $q \notin S$.  Throughout your mathematical upbringing, you have encountered several famous sets of numbers.  They are listed below.
\smallskip

\phantomsection
\label{setsofnumbersboxonthispage}

\colorbox{ResultColor}{\bbm

\centerline{\textbf{Sets of Numbers}}\index{set ! sets of numbers}

\begin{enumerate}

\item The \textbf{Empty Set}:\index{set ! empty}\index{empty set} $\emptyset=\{ \}=\{x\,|\,\mbox{$x
\neq x$}\}$.  This is the set with no elements.  Like the number `$0$,' it  plays a vital role in mathematics.\footnote{\ldots which, sadly, we will not explore in this text.}

\item The \textbf{Natural Numbers}:\index{natural number ! set of}\index{natural number ! definition of} $\mathbb N= \{ 1, 2, 3,  \ldots\}$ The periods of ellipsis here indicate that the natural numbers contain $1$, $2$, $3$, `and so forth'.

\item The \textbf{Whole Numbers}:\index{whole number ! set of}\index{whole number ! definition of} $\mathbb W = \{ 0, 1, 2, \ldots \}$

\item The \textbf{Integers}:\index{integer ! set of}\index{integer ! definition of} $\mathbb Z=\{ \ldots, -3, -2, -1, 0, 1, 2, 3, \ldots \}$

\item The \textbf{Rational Numbers}:\index{rational number ! set of}\index{rational number ! definition of} $\mathbb Q=\left\{\frac{a}{b} \, | \, a \in \mathbb Z \, \mbox{and} \, b \in \mathbb Z \right\}$.  \underline{Ratio}nal numbers are the \underline{ratio}s of integers (provided the denominator is not zero!)  It turns out that another way to describe the rational numbers\footnote{See Section \ref{Summation}.} is: \[\mathbb Q=\{x\,|\,\mbox{$x$ possesses a repeating or terminating decimal representation.}\}\]

\item The \textbf{Real Numbers}:\index{real number ! set of}\index{real number ! definition of} $\mathbb R = \{ x\,|\,\mbox{$x$ possesses a decimal representation.}\}$

\item The \textbf{Irrational Numbers}:\index{irrational number ! set of}\index{irrational number ! definition of} $\mathbb P = \{x\,|\,\mbox{$x$ is a non-rational real number.}\}$  Said another way, an \underline{ir}rational number is a decimal which neither repeats nor terminates.\footnote{The classic example is the number $\pi$ (See Section \ref{Angles}), but numbers like $\sqrt{2}$ and $0.101001000100001\ldots$ are other fine representatives.}

\item The \textbf{Complex Numbers}:\index{complex number ! set of}\index{complex number ! definition of} $\mathbb C=\{a+bi\,|\,\mbox{$a$,$b \in \mathbb R$ and $i=\sqrt{-1}$}\}$  Despite their importance, the complex numbers play only a minor role in the text.\footnote{They first appear in Section \ref{ComplexZeros} and return in Section \ref{PolarComplex}.} 

\end{enumerate}

\ebm}

\enlargethispage{.05in}
\smallskip
It is important to note that every natural number is a whole number, which, in turn, is an integer.   Each integer is a rational number (take $b =1$ in the above definition for $\mathbb Q$) and the rational numbers are all real numbers, since they possess decimal representations.\footnote{Long division, anyone?}   If we take $b=0$ in the above definition of $\mathbb C$, we see that every real number is a complex number.  In this sense, the sets $\mathbb N$, $\mathbb W$, $\mathbb Z$, $\mathbb Q$, $\mathbb R$, and $\mathbb C$ are `nested' like \href{http://en.wikipedia.org/wiki/Matryoshka_doll}{\underline{Matryoshka dolls}}.  

\bigskip

For the most part, this textbook focuses on sets whose elements come from the real numbers $\mathbb R$.  Recall that we may visualize $\mathbb R$ as a line. Segments of this line are called \textbf{intervals}\index{interval ! definition of} of numbers. Below is a summary of the so-called \textbf{interval notation}\index{interval ! notation for} associated with given sets of numbers.  For intervals with finite endpoints, we list the left endpoint, then the right endpoint.  We use square brackets, `$[$' or `$]$', if the endpoint is included in the interval and use a filled-in or `closed' dot to indicate membership in the interval. Otherwise, we use parentheses, `$($' or `$)$' and an `open' circle to indicate that the endpoint is not part of the set.  If the interval does not have finite endpoints, we use the symbols $-\infty$ to indicate that the interval extends indefinitely to the left and $\infty$ to indicate that the interval extends indefinitely to the right.  Since infinity is a concept, and not a number, we always use parentheses when using these symbols in interval notation, and use an appropriate arrow to indicate that the interval extends indefinitely in one (or both) directions.

\medskip

\colorbox{ResultColor}{\bbm

%\smallskip

\centerline{\textbf{Interval Notation}}

\medskip

\hspace{.5in} Let $a$ and $b$ be real numbers with $a<b$.

\smallskip

\begin{center}

\begin{tabular}{|c|c|c|} \hline

Set of Real Numbers & Interval Notation &  Region on the Real Number Line  \\
\hline

 &  & \\
\shortstack{$\{x\,|\,a<x<b\}$ \\ \hfill}& \shortstack{$(a,b)$ \\ \hfill} & 

\begin{mfpic}[10]{-3}{3}{-2}{2} 
\backgroundcolor[gray]{.95}

\tlpointsep{4pt}
\axislabels {x}{{$a\vphantom{b} \hspace{4pt} $} -3, {$b$} 3}

\polyline{(-3,0), (3,0)}
\pointfillfalse
\point[3pt]{(3,0), (-3,0)}

\end{mfpic}  \\ \hline

& &  \\
\shortstack{$\{x\,|\,a\leq x<b\}$ \\ \hfill}& \shortstack{$[a,b)$ \\ \hfill} & 

\begin{mfpic}[10]{-3}{3}{-2}{2} 
\backgroundcolor[gray]{.95}

\tlpointsep{4pt}
\axislabels {x}{{$a\vphantom{b} \hspace{4pt} $} -3, {$b$} 3}

\polyline{(-3,0), (3,0)}
\point[3pt]{(-3,0)}
\pointfillfalse
\point[3pt]{(3,0)}

\end{mfpic}   \\
\hline

 &  & \\
\shortstack{$\{x\,|\,a<x\leq b\}$ \\ \hfill}&\shortstack{$(a,b]$ \\ \hfill} & 

\begin{mfpic}[10]{-3}{3}{-2}{2} 
\backgroundcolor[gray]{.95}

\tlpointsep{4pt}
\axislabels {x}{{$a\vphantom{b} \hspace{4pt} $} -3, {$b$} 3}

\polyline{(-3,0), (3,0)}
\point[3pt]{(3,0)}
\pointfillfalse
\point[3pt]{(-3,0)}

\end{mfpic}   \\
\hline

 &  & \\
\shortstack{$\{x\,|\,a\leq x \leq b\}$ \\ \hfill}& \shortstack{$[a,b]$ \\ \hfill}& 

\begin{mfpic}[10]{-3}{3}{-2}{2} 
\backgroundcolor[gray]{.95}

\tlpointsep{4pt}
\axislabels {x}{{$a\vphantom{b} \hspace{4pt} $} -3, {$b$} 3}

\polyline{(-3,0), (3,0)}
\point[3pt]{(3,0), (-3,0)}

\end{mfpic}   \\
\hline

 & & \\
\shortstack{$\{x\,| \, x<b\}$ \\ \hfill}& \shortstack{$(-\infty,b)$ \\ \hfill}& 

\begin{mfpic}[10]{-3}{3}{-2}{2} 
\backgroundcolor[gray]{.95}

\tlpointsep{4pt}
\axislabels {x}{{$b$} 3}

\arrow \polyline{(3,0), (-3,0)}

\pointfillfalse
\point[3pt]{(3,0)}

\end{mfpic}   \\
\hline


&  & \\

\shortstack{$\{x\,| \, x \leq b\}$ \\ \hfill} & \shortstack{$(-\infty,b]$ \\ \hfill}& 

\begin{mfpic}[10]{-3}{3}{-2}{2} 
\backgroundcolor[gray]{.95}

\tlpointsep{4pt}
\axislabels {x}{{$b$} 3}

\arrow \polyline{(3,0), (-3,0)}

\point[3pt]{(3,0)}

\end{mfpic}   \\
\hline

 &  & \\
\shortstack{$\{x\,| \, x>a\}$ \\ \hfill}& \shortstack{$(a,\infty)$ \\ \hfill}& 

\begin{mfpic}[10]{-3}{3}{-2}{2} 
\backgroundcolor[gray]{.95}

\tlpointsep{4pt}
\axislabels {x}{{$a\vphantom{b} \hspace{4pt}$} -3}

\arrow \polyline{(-3,0), (3,0)}

\pointfillfalse
\point[3pt]{(-3,0)}

\end{mfpic}   \\
\hline

 &  & \\
\shortstack{$\{x\,| \, x \geq a \}$ \\ \hfill}& \shortstack{$[a,\infty)$ \\ \hfill} & 

\begin{mfpic}[10]{-3}{3}{-2}{2} 
\backgroundcolor[gray]{.95}

\tlpointsep{4pt}
\axislabels {x}{{$a\vphantom{b} \hspace{4pt}$} -3}

\arrow \polyline{(-3,0), (3,0)}

\point[3pt]{(-3,0)}

\end{mfpic}   \\
\hline

&  & \\
\shortstack{$\mathbb R$ \\ \hfill}& \shortstack{$(-\infty,\infty)$ \\ \hfill} & 

\begin{mfpic}[10]{-3}{3}{-2}{2} 
\backgroundcolor[gray]{.95}

\tlpointsep{4pt}
\axislabels {x}{{$\vphantom{b} \hspace{4pt}$} -3}

\arrow \reverse \arrow \polyline{(-3,0), (3,0)}

\end{mfpic}   \\
\hline

\end{tabular}

\end{center}

\ebm}

\pagebreak

For an example, consider the sets of real numbers described below.

\begin{center}
\begin{tabular}{|c|c|c|} \hline

Set of Real Numbers & Interval Notation &  Region on the Real Number Line  \\
\hline

& &  \\
\shortstack{$\{x\,|\,1\leq x< 3\}$ \\ \hfill} & \shortstack{$[1,3)$ \\ \hfill} & 

\begin{mfpic}[10]{-3}{3}{-2}{2} 


\tlpointsep{4pt}
\axislabels {x}{{$1 \hspace{4pt} $} -3, {$3$} 3}

\polyline{(-3,0), (3,0)}
\point[3pt]{(-3,0)}
\pointfillfalse
\point[3pt]{(3,0)}

\end{mfpic}   \\
\hline

 &  & \\
\shortstack{$\{x\,|\,-1\leq x \leq 4\}$ \\ \hfill}& \shortstack{$[-1,4]$ \\ \hfill} & 

\begin{mfpic}[10]{-3}{3}{-2}{2} 


\tlpointsep{4pt}
\axislabels {x}{{$-1 \hspace{8pt} $} -3, {$4$} 3}

\polyline{(-3,0), (3,0)}
\point[3pt]{(-3,0), (3,0)}

\end{mfpic}   \\
\hline

&  & \\

\shortstack{$\{x\,| \, x \leq 5 \}$ \\ \hfill} & \shortstack{$(-\infty, 5]$ \\ \hfill} &

\begin{mfpic}[10]{-3}{3}{-2}{2} 


\tlpointsep{4pt}
\axislabels {x}{{$5$} 3}

\arrow \polyline{(3,0), (-3,0)}
\point[3pt]{(3,0)}

\end{mfpic}   \\
\hline

 &  & \\
\shortstack{$\{x\,| \, x > -2 \}$ \\ \hfill} & \shortstack{$(-2, \infty)$ \\ \hfill} &  

\begin{mfpic}[10]{-3}{3}{-2}{2} 


\tlpointsep{4pt}
\axislabels {x}{{$-2 \hspace{8pt} $} -3}

\arrow \polyline{(-3,0), (3,0)}
\pointfillfalse
\point[3pt]{(-3,0)}

\end{mfpic}   \\
\hline

\end{tabular}

\end{center}

We will often have occasion to combine sets.  There are two basic ways to combine sets:  \textbf{intersection}
and \textbf{union}.  We define both of these concepts below.

\medskip

\colorbox{ResultColor}{\bbm

%\smallskip

\begin{defn} \label{intersectionunion}  Suppose $A$ and $B$ are two sets.

\begin{itemize}

\item The \textbf{intersection}\index{set ! intersection}\index{intersection of two sets} of $A$ and $B$:  $A \cap B = \{ x \, | \, x \in A \, \text{and} \,\, x \in B \}$

\item The \textbf{union}\index{set ! union}\index{union of two sets} of $A$ and $B$: $A \cup B = \{ x \, | \, x \in A \, \text{or} \,\, x \in B \, \, \text{(or both)} \}$

\end{itemize}

\end{defn}

\ebm}

\medskip

Said differently, the intersection of two sets is the overlap of the two sets -- the elements which the sets have in common.  The union of two sets consists of the totality of the elements in each of the sets, collected together.\footnote{The reader is encouraged to research \href{http://en.wikipedia.org/wiki/Venn_diagram}{\underline{\textbf{Venn Diagrams}}} for a nice geometric interpretation of these concepts.}  For example,  if $A = \{ 1,2,3 \}$ and $B = \{2,4,6 \}$, then $A \cap B = \{2\}$ and $A \cup B = \{1,2,3,4,6\}$.   If $A = [-5,3)$ and $B = (1, \infty)$, then we can find $A \cap B$ and $A\cup B$ graphically.  To find $A\cap B$, we shade  the overlap of the two and obtain $A \cap B = (1,3)$.  To find $A \cup B$, we shade each of $A$ and $B$ and describe the resulting shaded region to find  $A \cup B = [-5,\infty)$.

\begin{center}

\begin{tabular}{ccc}

\begin{mfpic}[10]{-5}{7}{-2}{2}
\arrow \reverse \arrow \polyline{(-5.5,0),(7.5,0)}
\axismarks{x}{-5,1,3}
\tlpointsep{4pt}
\axislabels {x}{{$-5 \hspace{8pt} $} -5, {$1$} 1,{$3$} 3,}
\penwd{1.15pt}
\polyline{(-5,2), (3,2)}
\arrow \polyline{ (1,1), (7,1)}
\point[4pt]{(-5,2)}
\pointfillfalse
\point[3pt]{(1,1), (3,2)}
\tcaption{$A = [-5,3)$,  $B = (1, \infty)$ }
\end{mfpic}  &

\begin{mfpic}[10]{-5}{7}{-2}{2}
\arrow \polyline{(-5,0),(7,0)}
\axismarks{x}{3}
\tlpointsep{5pt}
\axislabels {x}{{$-5 \hspace{8pt} $} -5, {$1$} 1,{$3$} 3,}
\penwd{1.15pt}
\polyline{(-5,2), (3,2)}
\arrow \polyline{ (1,1), (7,1)}
\polyline{(1,0), (3,0)}
\point[4pt]{(-5,2)}
\pointfillfalse
\point[3pt]{(1,0),(1,1), (3,2), (3,0)}
\tcaption{$A \cap B = (1,3)$}
\end{mfpic}  &

\begin{mfpic}[10]{-5}{7}{-2}{2}
\tlpointsep{5pt}
\axislabels {x}{{$-5 \hspace{8pt} $} -5, {$1$} 1,{$3$} 3,}
\penwd{1.15pt}
\polyline{(-5,2), (3,2)}
\arrow \polyline{ (1,1), (7,1)}
\arrow \polyline{(-5,0), (7,0)}
\point[4pt]{(-5,0),(-5,2)}
\pointfillfalse
\point[3pt]{(1,1), (3,2)}
\tcaption{$A \cup B = [-5,\infty)$}
\end{mfpic} \\


\end{tabular}

\end{center}

While both intersection and union are important, we have more occasion to use union in this text than intersection, simply because most of the sets of real numbers we will be working with are either intervals or are unions of intervals, as the following example illustrates.

\pagebreak

\begin{ex} \label{unionex} Express the following sets of numbers using interval notation.

\begin{multicols}{2}

\begin{enumerate}

\item  $\{ x \, | \, x \leq -2 \, \, \text{or} \, \,  x \geq 2 \}$

\item  $\{ x \, | \, x \neq 3 \}$

\setcounter{HW}{\value{enumi}}

\end{enumerate}

\end{multicols}

\begin{multicols}{2}

\begin{enumerate}

\setcounter{enumi}{\value{HW}}

\item  $\{ x \, | \, x \neq \pm 3 \}$

\item  $\{ x \, | \, -1 < x \leq 3 \,\, \text{or} \,\, x = 5\}$

\end{enumerate}

\end{multicols}

{\bf Solution.}

\begin{enumerate}

\item  The best way to proceed here is to graph the set of numbers on the number line and glean the answer from it.  The inequality $x \leq -2$ corresponds to the interval $(-\infty, -2]$ and the inequality $x \geq 2$ corresponds to the interval $[2, \infty)$.  Since we are looking to describe the real numbers $x$ in one of these \textit{or} the other, we have $\{ x \, | \, x \leq -2 \, \, \text{or} \, \,  x \geq 2 \} = (-\infty, -2] \cup [2, \infty)$.

\begin{center}

\begin{mfpic}[10]{-5}{5}{-2}{2}
\polyline{(-5,0), (5,0)}
\tlpointsep{5pt}
\axislabels {x}{{$-2 \hspace{8pt}$} -2, {$2$} 2}
\penwd{1.15pt}
\arrow \polyline{(2,0), (5,0)}
\arrow \polyline{(-2,0), (-5,0)}
\point[4pt]{(-2,0), (2,0)}
\end{mfpic}  \\
$(-\infty, -2] \cup [2, \infty)$ 

\end{center}

\item For the set $\{ x \, | \, x \neq 3 \}$, we shade the entire real number line except $x=3$, where we leave an open circle.  This divides the real number line into two intervals, $(-\infty, 3)$ and $(3,\infty)$.  Since the values of $x$ could be in either one of these intervals \textit{or} the other, we have that $\{ x \, | \, x \neq 3 \} = (-\infty, 3) \cup (3,\infty)$
 
\begin{center}

\begin{mfpic}[10]{-5}{5}{-2}{2}
\tlpointsep{5pt}
\axislabels {x}{{$3$} 0}
\penwd{1.15pt}
\arrow \reverse \arrow \polyline{(-5,0), (5,0)}
\pointfillfalse
\point[3pt]{(0,0)}
\end{mfpic}  \\



 $(-\infty, 3) \cup (3, \infty)$ 
 
 

\end{center}

\item  For the set $\{ x \, | \, x \neq \pm 3 \}$, we proceed as before and exclude both $x=3$ and $x=-3$ from our set.  This breaks the number line into \textit{three} intervals, $(-\infty, -3)$, $(-3,3)$ and $(3, \infty)$.   Since the set describes real numbers which come from the first, second \textit{or} third interval, we have $\{ x \, | \, x \neq \pm 3 \} = (-\infty, -3) \cup (-3,3) \cup (3, \infty)$.


\begin{center}

\begin{mfpic}[10]{-5}{5}{-2}{2}
\tlpointsep{5pt}
\axislabels {x}{{$-3 \hspace{8pt}$} -3, {$3$} 3}
\penwd{1.15pt}
\arrow \reverse \arrow \polyline{(-5,0), (5,0)}
\pointfillfalse
\point[3pt]{(-3,0), (3,0)}
\end{mfpic} \\

 $(-\infty, -3) \cup (-3,3) \cup (3, \infty)$
 
 \end{center}



\item  Graphing the set $\{ x \, | \, -1 < x \leq 3 \,\, \text{or} \,\, x = 5\}$, we get one interval, $(-1,3]$ along with a single number, or point, $\{ 5\}$.  While we \textit{could} express the latter as $[5,5]$ (Can you see why?), we choose to write our answer as $\{ x \, | \, -1 < x \leq 3 \,\, \text{or} \,\, x = 5\} = (-1,3] \cup \{ 5\}$.


\begin{center}

\begin{mfpic}[10]{-5}{5}{-2}{2}
\arrow \reverse \arrow \polyline{(-5,0), (5,0)}
\tlpointsep{5pt}
\axislabels {x}{{$-1 \hspace{8pt}$} -3, {$3$} 1, {$5$} 3}
\penwd{1.15pt}
\polyline{(-3,0), (1,0)}
\point[4pt]{(3,0), (1,0)}
\pointfillfalse
\point[3pt]{(-3,0)}
\end{mfpic} \\
  
 $(-1,3] \cup \{ 5\}$
 
\end{center}


\end{enumerate}

\vspace{-.5in}\qed

\end{ex}


\subsection{The Cartesian Coordinate Plane}

In order to visualize the pure excitement that is Precalculus, we need to unite Algebra and Geometry.  Simply put, we must find a way to draw algebraic things.  Let's start with possibly the greatest mathematical achievement of all time: the \index{Cartesian coordinate plane} \textbf{Cartesian Coordinate Plane}.\footnote{So named in honor of \href{http://en.wikipedia.org/wiki/Descartes}{\underline{Ren\'{e} Descartes}}.}  Imagine two real number lines crossing at a right angle at $0$ as drawn below.

\begin{center}

\begin{mfpic}[20]{-5}{5}{-5}{5}
\axes
\tlabel[cc](5,-0.5){\scriptsize $x$}
\tlabel[cc](0.5,5){\scriptsize $y$}
\xmarks{-4,-3,-2,-1,1,2,3,4}
\ymarks{-4,-3,-2,-1,1,2,3,4}
\tlpointsep{5pt}
\scriptsize
\axislabels {x}{{$-4 \hspace{7pt}$} -4, {$-3 \hspace{7pt}$} -3, {$-2 \hspace{7pt}$} -2, {$-1 \hspace{7pt}$} -1, {$1$} 1, {$2$} 2, {$3$} 3, {$4$} 4}
\axislabels {y}{{$-4$} -4, {$-3$} -3, {$-2$} -2, {$-1$} -1, {$1$} 1, {$2$} 2, {$3$} 3, {$4$} 4}
\normalsize
\end{mfpic}

\end{center}

\medskip

The horizontal number line is usually called the \index{$x$-axis} \textbf{\boldmath $x$-axis} while the vertical number line is usually called the \index{$y$-axis} \textbf{\boldmath $y$-axis}.\footnote{The labels can vary depending on the context of application.}  As with the usual number line, we imagine these axes extending off indefinitely in both directions.\footnote{Usually extending off  towards infinity is indicated by arrows, but here, the arrows are used to indicate the \textit{direction} of increasing values of $x$ and $y$.}
  Having two number lines allows us to locate the positions of points off of the number lines as well as points on the lines themselves.  

\medskip

For example, consider the point $P$ on the next page.  To use the numbers on the axes to label this point, we imagine dropping a vertical line from the $x$-axis to $P$ and extending a horizontal line from the $y$-axis to $P$.  This process is sometimes called `projecting' the point $P$ to the $x$- (respectively $y$-) axis.  We then describe the point $P$ using the \index{ordered pair} \textbf{ordered pair} $(2,-4)$.  The first number in the ordered pair is called the \index{abscissa} \textbf{abscissa} or \index{$x$-coordinate} \textbf{\boldmath $x$-coordinate} and the second is called the \index{ordinate} \textbf{ordinate} or \index{$y$-coordinate} \textbf{\boldmath $y$-coordinate}.\footnote{Again, the names of the coordinates can vary depending on the context of the application.  If, for example, the horizontal axis represented time we might choose to call it the $t$-axis.  The first number in the ordered pair would then be the $t$-coordinate.}  Taken together, the ordered pair $(2,-4)$ comprise the \index{coordinates ! Cartesian}\index{Cartesian coordinates}\textbf{Cartesian coordinates}\footnote{Also called the `rectangular coordinates' of $P$ -- see Section \ref{IntroPolar} for more details.} of the point $P$. In practice, the distinction between a point and its coordinates is blurred; for example, we often speak of `the point $(2,-4)$.'  We can think of $(2,-4)$ as instructions on how to reach $P$ from the \index{origin} {\bf origin} $(0, 0)$ by moving $2$ units to the right and $4$ units downwards.  Notice that the order in the \underline{ordered} pair is important $-$ if we wish to plot the point $(-4,2)$, we would move to the left $4$ units from the origin and then move upwards $2$ units, as below on the right.

\medskip

\hspace{.1in} \begin{tabular}{m{3in}m{3in}}
\begin{mfpic}[20]{-5}{5}{-5}{5}
\axes
\tlabel[cc](5,-0.5){\scriptsize $x$}
\tlabel[cc](0.5,5){\scriptsize $y$}
\xmarks{-4,-3,-2,-1,1,2,3,4}
\ymarks{-4,-3,-2,-1,1,2,3,4}
\gfill \circle{(2,-4),0.1}
\tlabel[cc](2.5,-4){\scriptsize $P$}
\dashed \polyline{(2,0),(2,-4),(0,-4)}
\tlpointsep{5pt}
\scriptsize
\axislabels {x}{{$-4 \hspace{7pt}$} -4, {$-3 \hspace{7pt} $} -3, {$-2\hspace{7pt} $} -2, {$-1 \hspace{7pt}$} -1, {$1$} 1, {$2$} 2, {$3$} 3, {$4$} 4}
\axislabels {y}{{$-4$} -4, {$-3$} -3, {$-2$} -2, {$-1$} -1, {$1$} 1, {$2$} 2, {$3$} 3, {$4$} 4}
\normalsize
\end{mfpic} &

\begin{mfpic}[20]{-5}{5}{-5}{5}
\axes
\tlabel[cc](5,-0.5){\scriptsize $x$}
\tlabel[cc](0.5,5){\scriptsize $y$}
\xmarks{-4,-3,-2,-1,1,2,3,4}
\ymarks{-4,-3,-2,-1,1,2,3,4}
\gfill \circle{(2,-4),0.1}
\tlabel[cc](3.5,-4){\scriptsize $P(2, -4)$}
\dashed \polyline{(2,0),(2,-4),(0,-4)}
\gfill \circle{(-4,2),0.1}
\tlabel[cc](-4,2.5){\scriptsize $(-4,2)$}
\dashed \polyline{(-4,0),(-4,2),(0,2)}
\tlpointsep{5pt}
\scriptsize
\axislabels {x}{{$-4 \hspace{7pt}$} -4, {$-3 \hspace{7pt}$} -3, {$-2 \hspace{7pt}$} -2, {$-1 \hspace{7pt}$} -1, {$1$} 1, {$2$} 2, {$3$} 3, {$4$} 4}
\axislabels {y}{{$-4$} -4, {$-3$} -3, {$-2$} -2, {$-1$} -1, {$1$} 1, {$2$} 2, {$3$} 3, {$4$} 4}
\end{mfpic} \\

\end{tabular}

When we speak of the Cartesian Coordinate Plane, we mean the set of all possible ordered pairs $(x,y)$ as $x$ and $y$ take values from the real numbers.  Below is a summary of important facts about Cartesian coordinates.

\smallskip 

\colorbox{ResultColor}{\bbm

%\smallskip

\centerline{\textbf{Important Facts about the Cartesian Coordinate Plane}}

\begin{itemize}

\item $(a,b)$ and $(c,d)$ represent the same point in the plane if and only if $a = c$ and $b = d$.

\item  $(x,y)$ lies on the $x$-axis if and only if $y = 0$.

\item  $(x,y)$ lies on the $y$-axis if and only if $x=0$.

\item The origin is the point $(0,0)$.  It is the only point common to both axes.

%\smallskip

\end{itemize}

\ebm}


\begin{ex} Plot the following points: $A(5,8)$, $B\left(-\frac{5}{2}, 3\right)$, $C(-5.8, -3)$, $D(4.5, -1)$, $E(5,0)$, $F(0,5)$, $G(-7,0)$, $H(0, -9)$, $O(0,0)$.\footnote{The letter $O$ is almost always reserved for the origin.}

\medskip

{\bf Solution.}  To plot these points, we start at the origin and move to the right if the $x$-coordinate is positive; to the left if it is negative.   Next, we move up if the $y$-coordinate is positive or down if it is negative.  If the $x$-coordinate is $0$, we start at the origin and move along the $y$-axis only.  If the  $y$-coordinate is $0$ we move along the $x$-axis only.


\begin{center}

\begin{mfpic}[16]{-10}{10}{-10}{10}
\axes
\tlabel[cc](10,-0.5){\scriptsize $x$}
\tlabel[cc](0.5,10){\scriptsize $y$}
\xmarks{-9,-8,-7,-6,-5,-4,-3,-2,-1,1,2,3,4,5,6,7,8,9}
\ymarks{-9,-8,-7,-6,-5,-4,-3,-2,-1,1,2,3,4,5,6,7,8,9}
\gfill \circle{(5,8),0.1}
\tlabel[cc](5,7.25){$A(5,8)$}
\gfill \circle{(-2.5,3),0.1}
\tlabel[cc](-2.5,2.25){$B\left(-\frac{5}{2},3\right)$}
\gfill \circle{(-5.8,-3),0.1}
\tlabel[cc](-5.8,-3.75){$C(-5.8,-3)$}
\gfill \circle{(4.5,-1),0.1}
\tlabel[cc](4.5,-1.75){$D(4.5,-1)$}
\gfill \circle{(5,0),0.1}
\tlabel[cc](5,0.5){$E(5,0)$}
\gfill \circle{(0,5),0.1}
\tlabel[cc](1.35,5){$F(0,5)$}
\gfill \circle{(-7,0),0.1}
\tlabel[cc](-7,0.5){$G(-7,0)$}
\gfill \circle{(0,-9),0.1}
\tlabel[cc](1.5,-9){$H(0,-9)$}
\gfill \circle{(0,0),0.1}
\tlabel[cc](1.2,0.5){$O(0,0)$}
\tlpointsep{5pt}
\scriptsize
\axislabels {x}{{$-9 \hspace{7pt}$} -9, {$-8 \hspace{7pt}$} -8, {$-7 \hspace{7pt}$} -7, {$-6 \hspace{7pt}$} -6, {$-5 \hspace{7pt}$} -5, {$-4 \hspace{7pt}$} -4, {$-3 \hspace{7pt}$} -3, {$-2 \hspace{7pt}$} -2, {$-1 \hspace{7pt}$} -1, {$1$} 1, {$2$} 2, {$3$} 3, {$4$} 4, {$5$} 5, {$6$} 6, {$7$} 7, {$8$} 8, {$9$} 9}
\axislabels {y}{{$-9$} -9, {$-8$} -8, {$-7$} -7, {$-6$} -6, {$-5$} -5, {$-4$} -4, {$-3$} -3, {$-2$} -2, {$-1$} -1, {$1$} 1, {$2$} 2, {$3$} 3, {$4$} 4, {$5$} 5, {$6$} 6, {$7$} 7, {$8$} 8, {$9$} 9}
\normalsize
\end{mfpic}

\end{center}

\qed

\end{ex}



The axes divide the plane into four regions called \index{quadrants} \textbf{quadrants}.  They are labeled with Roman numerals and proceed counterclockwise around the plane:

\label{quadrant}

\begin{center}
\begin{mfpic}[18]{-5}{5}{-5}{5}
\axes
\tlabel[cc](5,-0.5){\scriptsize $x$}
\tlabel[cc](0.5,5){\scriptsize $y$}
\tlabel[cc](3,3.5){Quadrant I}
\tlabel[cc](3,2.5){ $x > 0$, $y > 0$}
\tlabel[cc](-3,3.5){Quadrant II}
\tlabel[cc](-3,2.5){ $x < 0$, $y > 0$}
\tlabel[cc](-3,-2.5){Quadrant III}
\tlabel[cc](-3,-3.5){ $x < 0$, $y < 0$}
\tlabel[cc](3,-2.5){Quadrant IV}
\tlabel[cc](3,-3.5){ $x > 0$, $y < 0$}
\xmarks{-4,-3,-2,-1,1,2,3,4}
\ymarks{-4,-3,-2,-1,1,2,3,4}
\tlpointsep{5pt}
\scriptsize
\axislabels {x}{{$-4 \hspace{7pt}$} -4, {$-3 \hspace{7pt}$} -3, {$-2 \hspace{7pt}$} -2, {$-1 \hspace{7pt}$} -1, {$1$} 1, {$2$} 2, {$3$} 3, {$4$} 4}
\axislabels {y}{{$-4$} -4, {$-3$} -3, {$-2$} -2, {$-1$} -1, {$1$} 1, {$2$} 2, {$3$} 3, {$4$} 4}
\normalsize
\end{mfpic}

\end{center}

For example, $(1,2)$ lies in Quadrant I, $(-1,2)$ in Quadrant II, $(-1,-2)$ in Quadrant III and $(1,-2)$ in Quadrant IV.  If a point other than the origin happens to lie on the axes, we typically refer to that point as lying on the positive or negative $x$-axis (if $y = 0$) or on the positive or negative $y$-axis (if $x = 0$).  For example, $(0,4)$ lies on the positive $y$-axis whereas $(-117,0)$ lies on the negative $x$-axis.  Such points do not belong to any of the four quadrants.

\smallskip

One of the most important concepts in all of Mathematics is \textbf{symmetry}.\footnote{According to Carl.  Jeff thinks symmetry is overrated.}  There are many types of symmetry in Mathematics, but three of them can be discussed easily using Cartesian Coordinates.

\medskip

\colorbox{ResultColor}{\bbm

%\smallskip

\begin{defn}

\label{symmetrydefn}

Two points $(a,b)$ and $(c,d)$ in the plane are said to be

\begin{itemize}

\item \index{symmetry ! about the $x$-axis} \textbf{symmetric about the \boldmath $x$-axis} if $a = c$ and $b = -d$

\item \index{symmetry ! about the $y$-axis} \textbf{symmetric about the \boldmath $y$-axis} if $a = -c$ and $b = d$

\item \index{symmetry ! about the origin} \textbf{symmetric about the origin} if $a = -c$ and $b = -d$

\end{itemize}

\end{defn} 

\ebm}

\medskip

Schematically,

\begin{center}

\begin{mfpic}[15]{-5}{5}{-5}{5}
\axes
\tlabel[cc](0.25,-0.35){$0$}
\tlabel[cc](5,-0.5){\scriptsize $x$}
\tlabel[cc](0.5,5){\scriptsize $y$}
\gfill \circle{(4,2),0.1}
\tlabel[cc](4,3){$P(x,y)$}
\gfill \circle{(-4,2),0.1}
\tlabel[cc](-4,3){$Q(-x,y)$}
\gfill \circle{(4,-2),0.1}
\tlabel[cc](4,-3){$S(x,-y)$}
\gfill \circle{(-4,-2),0.1}
\tlabel[cc](-4,-3){$R(-x,-y)$}
\end{mfpic}

\end{center}

In the above figure, $P$ and $S$ are symmetric about the $x$-axis, as are $Q$ and $R$;  $P$ and $Q$ are symmetric about the $y$-axis, as are $R$ and $S$;  and $P$ and $R$ are symmetric about the origin, as are $Q$ and $S$.

\begin{ex}  Let $P$ be the point $(-2,3)$.  Find the points which are symmetric to $P$ about the:

\begin{multicols}{3}

\begin{enumerate}

\item  $x$-axis

\item  $y$-axis

\item  origin

\end{enumerate}

\end{multicols}

Check your answer by plotting the points.

\medskip

{\bf Solution.} The figure after Definition \ref{symmetrydefn} gives us a good way to think about finding symmetric points in terms of taking the opposites of the $x$- and/or $y$-coordinates of $P(-2,3)$.

\begin{enumerate}

\item  To find the point symmetric about the $x$-axis, we replace the $y$-coordinate with its opposite to get  $(-2,-3)$.

\item  To find the point symmetric about the $y$-axis, we replace the $x$-coordinate with its opposite to get $(2,3)$.

\item  To find the point symmetric about the origin, we replace the $x$- and $y$-coordinates with their opposites to get $(2,-3)$.

\end{enumerate}

\begin{center}

\begin{mfpic}[20]{-4}{4}{-4}{4}
\axes
\tlabel[cc](4.1,-0.5){\scriptsize $x$}
\tlabel[cc](0.5,4.1){\scriptsize $y$}
\gfill \circle{(-2,3),0.1}
\tlabel[cc](-2.5,2){\scriptsize $P(-2,3)$}
\gfill \circle{(-2,-3),0.1}
\tlabel[cc](-2.5,-3.7){\scriptsize $(-2,-3)$}
\gfill \circle{(2,3),0.1}
\tlabel[cc](2,2){\scriptsize $(2,3)$}
\gfill \circle{(2,-3),0.1}
\tlabel[cc](2,-3.7){\scriptsize $(2,-3)$}
\xmarks{-3,-2,-1,1,2,3}
\ymarks{-3,-2,-1,1,2,3}
\tlpointsep{5pt}
\scriptsize
\axislabels {x}{{$-3 \hspace{7pt}$} -3, {$-2 \hspace{7pt}$} -2, {$-1 \hspace{7pt}$} -1, {$1$} 1, {$2$} 2, {$3$} 3}
\axislabels {y}{{$-3$} -3, {$-2$} -2, {$-1$} -1, {$1$} 1, {$2$} 2, {$3$} 3}
\normalsize

\end{mfpic}

\end{center}

\vspace{-.4in}

\qed

\end{ex}

One way to visualize the processes in the previous example is with the concept of a \index{reflection ! of a point} \textbf{reflection}.  If we start with our point $(-2,3)$ and pretend that the $x$-axis is a mirror, then the reflection of $(-2,3)$ across the $x$-axis would lie at $(-2,-3)$.  If we pretend that the $y$-axis is a mirror, the reflection of $(-2,3)$ across that axis would be $(2,3)$.  If we reflect across the $x$-axis and then the $y$-axis, we would go from $(-2,3)$ to $(-2,-3)$ then to $(2,-3)$, and so we would end up at the point symmetric to $(-2,3)$ about the origin.  We summarize and generalize this process below.

\medskip

\colorbox{ResultColor}{\bbm

%\smallskip

\centerline{\textbf{Reflections}}

\hspace{.17in} To reflect a point $(x,y)$ about the:

\begin{itemize}

\item  $x$-axis, replace $y$ with $-y$.

\item  $y$-axis, replace $x$ with $-x$.

\item  origin, replace $x$ with $-x$ and $y$ with $-y$.

\end{itemize}

\ebm}

\subsection{Distance in the Plane}

Another important concept in Geometry is the notion of length.  If we are going to unite Algebra and Geometry using the Cartesian Plane, then we need to develop an algebraic understanding of what distance in the plane means.  Suppose we have two points, $P\left(x_{\mbox{\tiny$0$}}, y_{\mbox{\tiny$0$}}\right)$ and $Q\left(x_{\mbox{\tiny$1$}}, y_{\mbox{\tiny$1$}}\right),$ in the plane. By the \index{distance ! definition} \textbf{distance} $d$  between $P$ and $Q$, we mean the length of the line segment joining $P$ with $Q$.  (Remember, given any two distinct points in the plane, there is a unique line containing both points.)  Our goal now is to create an algebraic formula to compute the distance between these two points. Consider the generic situation below on the left.

\medskip

\hspace{.8in} \begin{tabular}{m{2.5in}m{2.75in}}

\begin{mfpic}[20]{-1}{5}{-1}{4}
\gfill \circle{(0,0),0.1}
\tlabel[c](-1,-1){$P\left(x_{\mbox{\tiny$0$}}, y_{\mbox{\tiny$0$}}\right)$}
\gfill \circle{(4,3),0.1}
\tlabel[c](4.25,3){$Q\left(x_{\mbox{\tiny$1$}}, y_{\mbox{\tiny$1$}}\right)$}
\arrow\reverse\arrow \polyline{(0.1,0.075), (3.9,2.925)}
\tlabel[c](1.25,2.25){$d$}
\end{mfpic} & 
\begin{mfpic}[20]{-1}{5}{-1}{4}
\gfill \circle{(0,0),0.1}
\tlabel[c](-1,-1){$P\left(x_{\mbox{\tiny$0$}}, y_{\mbox{\tiny$0$}}\right)$}
\gfill \circle{(4,3),0.1}
\tlabel[c](4.25,3){$Q\left(x_{\mbox{\tiny$1$}}, y_{\mbox{\tiny$1$}}\right)$}
\arrow\reverse\arrow \polyline{(0.1,0.075), (3.9,2.925)}
\tlabel[c](1.25,2.25){$d$}
\dashed \polyline{(0,0), (4,0), (4,3)}
\gfill \circle{(4,0),0.1}
\tlabel[c](4,-1){$\left(x_{\mbox{\tiny$1$}}, y_{\mbox{\tiny$0$}}\right)$}
\polyline{(3.5, 0), (3.5, 0.5), (4, 0.5)}
\end{mfpic} \\ 

\end{tabular}

\medskip

With a little more imagination, we can envision a right triangle whose hypotenuse has length $d$ as drawn above on the right.  From the latter figure, we see that the lengths of the legs of the triangle are $\left|x_{\mbox{\tiny$1$}} - x_{\mbox{\tiny$0$}}\right|$ and $\left|y_{\mbox{\tiny$1$}} - y_{\mbox{\tiny$0$}}\right|$ so the \href{http://en.wikipedia.org/wiki/Pythagorean_Theorem}{\underline{Pythagorean Theorem}} gives us
 
 \[ \left|x_{\mbox{\tiny$1$}} - x_{\mbox{\tiny$0$}}\right|^2 + \left|y_{\mbox{\tiny$1$}} - y_{\mbox{\tiny$0$}}\right|^2 = d^2\]
 \[ \left(x_{\mbox{\tiny$1$}} - x_{\mbox{\tiny$0$}}\right)^2 + \left(y_{\mbox{\tiny$1$}} - y_{\mbox{\tiny$0$}}\right)^2 = d^2\]
 
(Do you remember why we can replace the absolute value notation with parentheses?)  By extracting the square root of both sides of the second equation and using the fact that distance is never negative, we get
 
\medskip
 
\colorbox{ResultColor}{\bbm

%\smallskip

\begin{eqn} \label{distanceformula}\index{distance ! distance formula}\textbf{The Distance Formula:}  The distance $d$ between the points $P\left(x_{\mbox{\tiny$0$}}, y_{\mbox{\tiny$0$}}\right)$ and $Q\left(x_{\mbox{\tiny$1$}}, y_{\mbox{\tiny$1$}}\right)$ is:
 
\[d = \sqrt{ \left(x_{\mbox{\tiny$1$}} - x_{\mbox{\tiny$0$}}\right)^2 + \left(y_{\mbox{\tiny$1$}} - y_{\mbox{\tiny$0$}}\right)^2} \]

\end{eqn}

\ebm}

\medskip

It is not always the case that the points $P$ and $Q$ lend themselves to constructing such a triangle.  If the points $P$ and $Q$ are arranged vertically or horizontally, or describe the exact same point, we cannot use the above geometric argument to derive the distance formula.  It is left to the reader in Exercise \ref{distanceothercases} to verify Equation \ref{distanceformula} for these cases.

\begin{ex}  Find and simplify the distance between $P(-2,3)$ and  $Q(1,-3)$.  

\medskip

\flushleft {\bf Solution.}

\setlength{\extrarowheight}{3pt}

\[ \begin{array}{rcl}

 d & = & \sqrt{\left(x_{\mbox{\tiny$1$}} - x_{\mbox{\tiny$0$}} \right)^2 + \left(y_{\mbox{\tiny$1$}} - y_{\mbox{\tiny$0$}} \right)^2} \\
   & = & \sqrt{ (1-(-2))^2 + (-3-3)^2} \\
   & = & \sqrt{9 + 36} \\
   & = & 3 \sqrt{5} \end{array} \]

\setlength{\extrarowheight}{2pt}

\medskip

So the distance is $3 \sqrt{5}$. \qed

\end{ex}

\begin{ex}  Find all of the points with $x$-coordinate $1$ which are $4$ units from the point $(3,2)$.

\medskip

\flushleft {\bf Solution.}  We shall soon see that the points we wish to find are on the line $x=1$, but for now we'll just view them as points of the form $(1,y)$.  Visually,

\begin{center}

\begin{mfpic}[20]{-1}{4}{-4}{4}
\axes
\arrow \reverse \arrow \polyline{(1,-3.5),(1,3.5)}
\gfill \circle{(3,2),0.1}
\gfill \circle{(1,-1.5),0.1}
\tlabel(1.45,-1.65){\footnotesize $(1,y)$}
\dashed \arrow \reverse \arrow \polyline{(1.1,-1.325),(2.9,1.825)}
\tlabel(3.25,1.9){\footnotesize $(3,2)$}
\tlabel(4,-0.25){ \footnotesize $x$}
\tlabel(0.25,4){ \footnotesize $y$}
\tlabel(2.5,0.5){\footnotesize distance is 4 units}
\xmarks{0,1,2,3}
\ymarks{-3,-2,-1,0,1,2,3}
\tlpointsep{5pt}
\scriptsize
\axislabels {x}{{$2$} 2, {$3$} 3}
\axislabels {y}{{$-3$} -3, {$-2$} -2, {$-1$} -1, {$1$} 1, {$2$} 2, {$3$} 3}
\normalsize
\end{mfpic}

\end{center}

We require that the distance from $(3,2)$ to $(1,y)$ be $4$.  The Distance Formula, Equation \ref{distanceformula}, yields

\[ \begin{array}{rclr} 
d &  = & \sqrt{\left(x_{\mbox{\tiny$1$}}-x_{\mbox{\tiny$0$}}\right)^2+\left(y_{\mbox{\tiny$1$}}-y_{\mbox{\tiny$0$}}\right)^2}  & \\
4 &  = & \sqrt{(1-3)^2+(y-2)^2} & \\
4  & = & \sqrt{4+(y-2)^2} & \\ 
4^2 & = & \left(\sqrt{4+(y-2)^2}\right)^2 &  \mbox{squaring both sides} \\
16 & = & 4+(y-2)^2 & \\
12 & = & (y-2)^2 & \\
(y-2)^2 & = & 12 &  \\
y - 2 & = & \pm \sqrt{12} & \mbox{extracting the square root} \\
y-2 & = & \pm 2 \sqrt{3} & \\
y & = & 2 \pm 2 \sqrt{3}  & 
\end{array} \]


We obtain two answers:  $(1, 2 + 2 \sqrt{3})$ and $(1, 2-2 \sqrt{3}).$  The reader is encouraged to think about why there are two answers. \qed

\end{ex}

Related to finding the distance between two points is the problem of finding the \index{midpoint ! definition of} \textbf{midpoint} of the line segment connecting two points.  Given two points, $P\left(x_{\mbox{\tiny$0$}}, y_{\mbox{\tiny$0$}}\right)$ and $Q\left(x_{\mbox{\tiny$1$}}, y_{\mbox{\tiny$1$}}\right)$, the \textbf{midpoint} $M$  of $P$ and $Q$ is defined to be the point on the line segment connecting $P$ and $Q$ whose distance from $P$ is equal to its distance from  $Q$.  

 \begin{center}

\begin{mfpic}[15]{-1}{5}{-1}{4}
\gfill \circle{(0,0),0.1}
\tlabel[c](-1,-0.9){$P\left(x_{\mbox{\tiny$0$}}, y_{\mbox{\tiny$0$}}\right)$}
\gfill \circle{(4,3),0.1}
\tlabel[c](4.25,3){$Q\left(x_{\mbox{\tiny$1$}}, y_{\mbox{\tiny$1$}}\right)$}
\polyline{(0,0), (4,3)}
\gfill \circle{(2,1.5),0.1}
\tlabel[c](2,1){$M$}
\end{mfpic}

\end{center}

If we think of reaching $M$ by going `halfway over' and `halfway up' we get the following formula. 

\medskip

\colorbox{ResultColor}{\bbm

%\smallskip

\begin{eqn} \index{midpoint ! midpoint formula}\label{midpointformula}\textbf{The Midpoint Formula:}  The midpoint $M$ of the line segment connecting $P\left(x_{\mbox{\tiny$0$}}, y_{\mbox{\tiny$0$}}\right)$ and $Q\left(x_{\mbox{\tiny$1$}}, y_{\mbox{\tiny$1$}}\right)$ is:

\[ M = \left( \dfrac{x_{\mbox{\tiny$0$}} + x_{\mbox{\tiny$1$}}}{2} , \dfrac{y_{\mbox{\tiny$0$}} + y_{\mbox{\tiny$1$}}}{2} \right)\]

\end{eqn}

\ebm}

\medskip

If we let $d$ denote the distance between $P$ and $Q$, we leave it as Exercise \ref{verifymidpointformula} to show that the distance between $P$ and $M$ is $d/2$ which is the same as the distance between $M$ and $Q$.  This suffices to show that Equation \ref{midpointformula} gives the coordinates of the midpoint.

\begin{ex} 

Find the midpoint of the line segment connecting $P(-2,3)$ and  $Q(1,-3)$.  

\medskip

{\bf Solution.}

\setlength{\extrarowheight}{10pt}

\[ \begin{array}{rcl}
 M & = & \left( \dfrac{x_{\mbox{\tiny$0$}}+x_{\mbox{\tiny$1$}}}{2},  \dfrac{y_{\mbox{\tiny$0$}}+y_{\mbox{\tiny$1$}}}{2} \right) \\
   & = & \left( \dfrac{(-2)+1}{2},  \dfrac{3+(-3)}{2} \right)  = \left(- \dfrac{1}{2}, \dfrac{0}{2} \right) \\
   & = & \left(- \dfrac{1}{2}, 0 \right) 
   \end{array} \]
   
The midpoint is  $\left(- \frac{1}{2}, 0 \right)$.  \qed

\end{ex}

\phantomsection

\label{inversemidpoint}

We close with a more abstract application of the Midpoint Formula.  We will revisit the following example in Exercise \ref{inversemidpointex2} in Section \ref{LinearFunctions}.  

\begin{ex} \label{inversemidpointex1} If $a \neq b$, prove that the line $y = x$ equally divides the line segment with endpoints $(a,b)$ and $(b,a)$.

\medskip

{\bf Solution.}  To prove the claim, we use Equation \ref{midpointformula} to find the midpoint  

\setlength{\extrarowheight}{10pt}

\[ \begin{array}{rcl}

 M & = & \left( \dfrac{a+b}{2},  \dfrac{b+a}{2} \right) \\
   & = & \left( \dfrac{a+b}{2},  \dfrac{a+b}{2} \right)  \\ \end{array} \]

Since the $x$ and $y$ coordinates of this point are the same, we find that the midpoint lies on the line $y=x$, as required. \qed

\end{ex}

\setlength{\extrarowheight}{2pt}

\newpage

\subsection{Exercises}

\begin{enumerate}

\item Fill in the chart below:

\begin{center}
\begin{tabular}{|c|c|c|} \hline

Set of Real Numbers & Interval Notation &  Region on the Real Number Line  \\
\hline

& &  \\

\shortstack{$\{x\,|\,-1\leq x< 5\}$ \\ \hfill} &  &  \\ \hline

& &  \\

 & \shortstack{$[0,3)$ \\ \hfill} &   \\ \hline


& &  \\

 &  & 

\begin{mfpic}[10]{-3}{3}{-2}{2} 
\tlpointsep{4pt}
\axislabels {x}{{$2 \hspace{4pt} $} -3, {$7$} 3}
\polyline{(-3,0), (3,0)}
\point[3pt]{(3,0)}
\pointfillfalse
\point[3pt]{(-3,0)}
\end{mfpic}   \\
\hline

 &  & \\
 
\shortstack{$\{x\,|\, -5 <  x \leq 0 \}$ \\ \hfill} &  & \\ \hline

 &  & \\
 
  & \shortstack{$(-3,3)$ \\ \hfill} &  \\ \hline

&  & \\
 
& & 

\begin{mfpic}[10]{-3}{3}{-2}{2} 
\tlpointsep{4pt}
\axislabels {x}{{$5 \hspace{4pt} $} -3, {$7$} 3}
\polyline{(-3,0), (3,0)}
\point[3pt]{(-3,0), (3,0)}

\end{mfpic}   \\
\hline

&  & \\

\shortstack{$\{x\,| \, x \leq 3 \}$ \\ \hfill} &  &  \\ \hline

 &  & \\
 
& \shortstack{$(-\infty, 9)$ \\ \hfill} &  \\ \hline

 &  & \\

 &  &  

\begin{mfpic}[10]{-3}{3}{-2}{2} 
\tlpointsep{4pt}
\axislabels {x}{{$4 \hspace{4pt} $} -3}
\arrow \polyline{(-3,0), (3,0)}
\pointfillfalse
\point[3pt]{(-3,0)}

\end{mfpic}   \\
\hline

 &  & \\
 
 
\shortstack{$\{x\,| \, x \geq  -3 \}$ \\ \hfill} & &    \\ \hline

\end{tabular}

\end{center}

\setcounter{HW}{\value{enumi}}
\end{enumerate}

In Exercises \ref{findunionintfirst} - \ref{findunionintlast}, find the indicated intersection or union and simplify if possible.  Express your answers in interval notation.

\begin{multicols}{3}
\begin{enumerate}
\setcounter{enumi}{\value{HW}}

\item  $(-1,5] \cap [0,8)$ \label{findunionintfirst}
\item  $(-1,1) \cup [0,6]$
\item $(-\infty,4]\cap (0,\infty)$

\setcounter{HW}{\value{enumi}}
\end{enumerate}
\end{multicols}

\begin{multicols}{3}
\begin{enumerate}
\setcounter{enumi}{\value{HW}}

\item $(-\infty,0) \cap [1,5]$
\item $(-\infty, 0) \cup [1,5]$
\item $(-\infty, 5] \cap [5,8)$ \label{findunionintlast}

\setcounter{HW}{\value{enumi}}
\end{enumerate}
\end{multicols}

In Exercises \ref{writeintervalfirst} - \ref{writeintervallast}, write the set using interval notation.

\begin{multicols}{3}
\begin{enumerate}
\setcounter{enumi}{\value{HW}}

\item  $\{x\,|\, x \neq 5 \}$ \label{writeintervalfirst}
\item  $\{x\,|\, x \neq -1 \}$
\item  $\{x\,|\, x \neq -3,\, 4 \}$

\setcounter{HW}{\value{enumi}}
\end{enumerate}
\end{multicols}

\begin{multicols}{3}
\begin{enumerate}
\setcounter{enumi}{\value{HW}}

\item  $\{x\,|\, x \neq 0, \, 2 \}$
\item  $\{x\,|\, x \neq 2, \, -2 \}$
\item  $\{x\,|\, x \neq 0,\, \pm 4 \}$

\setcounter{HW}{\value{enumi}}
\end{enumerate}
\end{multicols}

\begin{multicols}{3}
\begin{enumerate}
\setcounter{enumi}{\value{HW}}

\item $\{x\,|\, x \leq -1 \, \text{or} \, x \geq 1 \}$
\item $\{x\,|\, x < 3 \, \text{or} \, x \geq 2 \}$
\item $\{x\,|\, x \leq -3 \, \text{or} \, x > 0 \}$

\setcounter{HW}{\value{enumi}}
\end{enumerate}
\end{multicols}

\begin{multicols}{3}
\begin{enumerate}
\setcounter{enumi}{\value{HW}}

\item $\{x\,|\, x \leq 5 \, \text{or} \, x = 6 \}$
\item $\{x\,|\, x > 2 \, \text{or} \, x = \pm 1 \}$
\item $\{x\,|\,  -3 < x < 3 \, \text{or} \, x = 4 \}$ \label{writeintervallast}

\setcounter{HW}{\value{enumi}}
\end{enumerate}
\end{multicols}


\begin{enumerate}
\setcounter{enumi}{\value{HW}}

\item Plot and label the points $\;A(-3, -7)$,  $\;B(1.3, -2)$,  $\;C(\pi, \sqrt{10})$,  $\;D(0, 8)$,  $\;E(-5.5, 0)$,  $\;F(-8, 4)$, $\;G(9.2, -7.8)$ and $H(7, 5)$ in the Cartesian Coordinate Plane given below. 
\label{cartexerciseone}

\begin{center}

\begin{mfpic}[15]{-10}{10}{-10}{10}
\axes
\tlabel[cc](10,-0.5){\scriptsize $x$}
\tlabel[cc](0.5,10){\scriptsize $y$}
\xmarks{-9,-8,-7,-6,-5,-4,-3,-2,-1,1,2,3,4,5,6,7,8,9}
\ymarks{-9,-8,-7,-6,-5,-4,-3,-2,-1,1,2,3,4,5,6,7,8,9}
\tlpointsep{5pt}
\scriptsize
\axislabels {x}{{$-9 \hspace{7pt}$} -9, {$-8 \hspace{7pt}$} -8, {$-7 \hspace{7pt}$} -7, {$-6 \hspace{7pt}$} -6, {$-5 \hspace{7pt}$} -5, {$-4 \hspace{7pt}$} -4, {$-3 \hspace{7pt}$} -3, {$-2 \hspace{7pt}$} -2, {$-1 \hspace{7pt}$} -1, {$1$} 1, {$2$} 2, {$3$} 3, {$4$} 4, {$5$} 5, {$6$} 6, {$7$} 7, {$8$} 8, {$9$} 9}
\axislabels {y}{{$-9$} -9, {$-8$} -8, {$-7$} -7, {$-6$} -6, {$-5$} -5, {$-4$} -4, {$-3$} -3, {$-2$} -2, {$-1$} -1, {$1$} 1, {$2$} 2, {$3$} 3, {$4$} 4, {$5$} 5, {$6$} 6, {$7$} 7, {$8$} 8, {$9$} 9}
\normalsize
\end{mfpic}

\end{center}

\item For each point given in Exercise \hspace{-.1in} ~\ref{cartexerciseone} above

\begin{itemize}

\item Identify the quadrant or axis in/on which the point lies.
\item Find the point symmetric to the given point about the $x$-axis.
\item Find the point symmetric to the given point about the $y$-axis.
\item Find the point symmetric to the given point about the origin.

\end{itemize}

\setcounter{HW}{\value{enumi}}

\end{enumerate}

\pagebreak

In Exercises \ref{distmidfirst} - \ref{distmidlast}, find the distance $d$ between the points and the midpoint $M$ of the line segment which connects them.

\begin{multicols}{2}
\begin{enumerate}
\setcounter{enumi}{\value{HW}}

\item $(1,2)$, $(-3,5)$ \label{distmidfirst}
\item $(3, -10)$, $(-1, 2)$ 

\setcounter{HW}{\value{enumi}}
\end{enumerate}
\end{multicols}

\begin{multicols}{2}
\begin{enumerate}
\setcounter{enumi}{\value{HW}}

\item $\left( \dfrac{1}{2}, 4\right)$, $\left(\dfrac{3}{2}, -1\right)$ 
\item $\left(- \dfrac{2}{3}, \dfrac{3}{2} \right)$, $\left(\dfrac{7}{3}, 2\right)$ 

\setcounter{HW}{\value{enumi}}
\end{enumerate}
\end{multicols}


\begin{multicols}{2}
\begin{enumerate}
\setcounter{enumi}{\value{HW}}

\item  $\left( \dfrac{24}{5}, \dfrac{6}{5} \right)$, $\left( -\dfrac{11}{5}, -\dfrac{19}{5} \right)$.
\item $\left(\sqrt{2}, \sqrt{3}\right)$, $\left(-\sqrt{8}, -\sqrt{12}\right)$ \vphantom{$\left( \dfrac{6}{5} \right)$}

\setcounter{HW}{\value{enumi}}
\end{enumerate}
\end{multicols}

\begin{multicols}{2}
\begin{enumerate}
\setcounter{enumi}{\value{HW}}

\item  $\left(2 \sqrt{45}, \sqrt{12} \right)$, $\left(\sqrt{20}, \sqrt{27} \right)$.
\item $(0, 0)$, $(x, y)$ \label{distmidlast}

\setcounter{HW}{\value{enumi}}
\end{enumerate}
\end{multicols}

\begin{enumerate}
\setcounter{enumi}{\value{HW}}

\item Find all of the points of the form $(x, -1)$ which are $4$ units from the point $(3,2)$.

\item Find all of the points on the $y$-axis which are $5$ units from the point $(-5,3)$.

\item Find all of the points on the $x$-axis which are $2$ units from the point $(-1,1)$.

\item Find all of the points of the form $(x,-x)$ which are $1$ unit from the origin.

\item Let's assume for a moment that we are standing at the origin and the positive $y$-axis points due North while the positive $x$-axis points due East.  Our Sasquatch-o-meter tells us that Sasquatch is 3 miles West and 4 miles South of our current position.  What are the coordinates of his position?  How far away is he from us?  If he runs 7 miles due East what would his new position be?

\item \label{distanceothercases} Verify the Distance Formula \ref{distanceformula} for the cases when:

\begin{enumerate}

\item The points are arranged vertically.  (Hint: Use $P(a, y_{\mbox{\tiny$0$}})$ and $Q(a, y_{\mbox{\tiny$1$}})$.)
\item The points are arranged horizontally. (Hint: Use $P(x_{\mbox{\tiny$0$}}, b)$ and $Q(x_{\mbox{\tiny$1$}}, b)$.)
\item The points are actually the same point. (You shouldn't need a hint for this one.)

\end{enumerate}

\item \label{verifymidpointformula} Verify the Midpoint Formula by showing the distance between $P(x_{\mbox{\tiny$1$}}, y_{\mbox{\tiny$1$}})$ and $M$ and the distance between $M$ and $Q(x_{\mbox{\tiny$2$}}, y_{\mbox{\tiny$2$}})$ are both half of the distance between $P$ and $Q$. 

\item Show that the points $A$, $\;B$ and $C$ below are the vertices of a right triangle.

\begin{multicols}{2}

\begin{enumerate}

\item  $A(-3,2)$, $\;B(-6,4)$, and $C(1,8)$

\item   $A(-3, 1)$, $\;B(4, 0)$ and $C(0, -3)$


\end{enumerate}
\end{multicols}

\item Find a point $D(x, y)$ such that the points $A(-3, 1)$, $\;B(4, 0)$, $\;C(0, -3)$ and $D$ are the corners of a square.  Justify your answer.

\item Discuss with your classmates how many numbers are in the interval $(0,1)$.
\enlargethispage{.4in}
\item \label{orderedtripleexercise} The world is not flat.\footnote{There are those who disagree with this statement.  Look them up on the Internet some time when you're bored.}  Thus the Cartesian Plane cannot possibly be the end of the story.  Discuss with your classmates how you would extend Cartesian Coordinates to represent the three dimensional world.  What would the Distance and Midpoint formulas look like, assuming those concepts make sense at all?

\end{enumerate}

\newpage

\subsection{Answers}

\begin{enumerate}

\item $~$

\begin{center}
\begin{tabular}{|c|c|c|} \hline

Set of Real Numbers & Interval Notation &  Region on the Real Number Line  \\
\hline

& &  \\

\shortstack{$\{x\,|\,-1\leq x< 5\}$ \\ \hfill} & \shortstack{$[-1,5)$ \\ \hfill} & 

\begin{mfpic}[10]{-3}{3}{-2}{2} 
\tlpointsep{4pt}
\axislabels {x}{{$-1 \hspace{8pt} $} -3, {$5$} 3}
\polyline{(-3,0), (3,0)}
\point[3pt]{(-3,0)}
\pointfillfalse
\point[3pt]{(3,0)}
\end{mfpic}   \\
\hline

& &  \\

\shortstack{$\{x\,|\,0\leq x < 3\}$ \\ \hfill} & \shortstack{$[0,3)$ \\ \hfill} & 

\begin{mfpic}[10]{-3}{3}{-2}{2} 
\tlpointsep{4pt}
\axislabels {x}{{$0 \hspace{4pt} $} -3, {$3$} 3}
\polyline{(-3,0), (3,0)}
\point[3pt]{(-3,0)}
\pointfillfalse
\point[3pt]{(3,0)}
\end{mfpic}   \\
\hline


& &  \\

\shortstack{$\{x\,|\, 2 <  x \leq 7 \}$ \\ \hfill} & \shortstack{$(2,7]$ \\ \hfill} & 

\begin{mfpic}[10]{-3}{3}{-2}{2} 
\tlpointsep{4pt}
\axislabels {x}{{$2 \hspace{4pt} $} -3, {$7$} 3}
\polyline{(-3,0), (3,0)}
\point[3pt]{(3,0)}
\pointfillfalse
\point[3pt]{(-3,0)}
\end{mfpic}   \\
\hline

 &  & \\
 
 \shortstack{$\{x\,|\, -5 <  x \leq 0 \}$ \\ \hfill} & \shortstack{$(-5,0]$ \\ \hfill} & 

\begin{mfpic}[10]{-3}{3}{-2}{2} 
\tlpointsep{4pt}
\axislabels {x}{{$-5 \hspace{8pt} $} -3, {$0$} 3}
\polyline{(-3,0), (3,0)}
\point[3pt]{(3,0)}
\pointfillfalse
\point[3pt]{(-3,0)}
\end{mfpic}   \\
\hline

 &  & \\
 
 \shortstack{$\{x\,|\, -3 <  x < 3 \}$ \\ \hfill} & \shortstack{$(-3,3)$ \\ \hfill} & 

\begin{mfpic}[10]{-3}{3}{-2}{2} 
\tlpointsep{4pt}
\axislabels {x}{{$-3 \hspace{8pt} $} -3, {$3$} 3}
\polyline{(-3,0), (3,0)}
\pointfillfalse
\point[3pt]{(-3,0), (3,0)}
\end{mfpic}   \\
\hline

 &  & \\
 
\shortstack{$\{x\,|\,5\leq x \leq 7\}$ \\ \hfill}& \shortstack{$[5,7]$ \\ \hfill} & 

\begin{mfpic}[10]{-3}{3}{-2}{2} 
\tlpointsep{4pt}
\axislabels {x}{{$5 \hspace{4pt} $} -3, {$7$} 3}
\polyline{(-3,0), (3,0)}
\point[3pt]{(-3,0), (3,0)}

\end{mfpic}   \\
\hline

&  & \\

\shortstack{$\{x\,| \, x \leq 3 \}$ \\ \hfill} & \shortstack{$(-\infty, 3]$ \\ \hfill} &
\begin{mfpic}[10]{-3}{3}{-2}{2} 
\tlpointsep{4pt}
\axislabels {x}{{$3$} 3}
\arrow \polyline{(3,0), (-3,0)}
\point[3pt]{(3,0)}

\end{mfpic}   \\
\hline

 &  & \\
 
 \shortstack{$\{x\,| \, x < 9 \}$ \\ \hfill} & \shortstack{$(-\infty, 9)$ \\ \hfill} &
\begin{mfpic}[10]{-3}{3}{-2}{2} 
\tlpointsep{4pt}
\axislabels {x}{{$9$} 3}
\arrow \polyline{(3,0), (-3,0)}
\pointfillfalse
\point[3pt]{(3,0)}

\end{mfpic}   \\
\hline

 &  & \\
 
 
\shortstack{$\{x\,| \, x >  4 \}$ \\ \hfill} & \shortstack{$(4, \infty)$ \\ \hfill} &  

\begin{mfpic}[10]{-3}{3}{-2}{2} 
\tlpointsep{4pt}
\axislabels {x}{{$4 \hspace{4pt} $} -3}
\arrow \polyline{(-3,0), (3,0)}
\pointfillfalse
\point[3pt]{(-3,0)}

\end{mfpic}   \\
\hline

 &  & \\
 
 
\shortstack{$\{x\,| \, x \geq  -3 \}$ \\ \hfill} & \shortstack{$[-3, \infty)$ \\ \hfill} &  

\begin{mfpic}[10]{-3}{3}{-2}{2} 
\tlpointsep{4pt}
\axislabels {x}{{$-3 \hspace{8pt} $} -3}
\arrow \polyline{(-3,0), (3,0)}
\point[3pt]{(-3,0)}

\end{mfpic}   \\
\hline

\end{tabular}

\end{center}

\setcounter{HW}{\value{enumi}}
\end{enumerate}

\begin{multicols}{2}
\begin{enumerate}
\setcounter{enumi}{\value{HW}}

\item  $(-1,5] \cap [0,8) = [0,5]$

\item  $(-1,1) \cup [0,6] = (-1,6]$

\setcounter{HW}{\value{enumi}}
\end{enumerate}
\end{multicols}

\begin{multicols}{2}
\begin{enumerate}
\setcounter{enumi}{\value{HW}}

\item $(-\infty,4]\cap (0,\infty) = (0,4]$


\item $(-\infty,0) \cap [1,5] = \emptyset$

\setcounter{HW}{\value{enumi}}
\end{enumerate}
\end{multicols}

\begin{multicols}{2}
\begin{enumerate}
\setcounter{enumi}{\value{HW}}

\item $(-\infty, 0) \cup [1,5] = (-\infty,0) \cup [1,5]$

\item $(-\infty, 5] \cap [5,8) = \left\{ 5\right\}$

\setcounter{HW}{\value{enumi}}
\end{enumerate}
\end{multicols}

\begin{multicols}{2}
\begin{enumerate}
\setcounter{enumi}{\value{HW}}

\item  $(-\infty, 5) \cup (5, \infty)$

\item  $(-\infty, -1) \cup (-1, \infty)$

\setcounter{HW}{\value{enumi}}
\end{enumerate}
\end{multicols}


\begin{multicols}{2}
\begin{enumerate}
\setcounter{enumi}{\value{HW}}

\item  $(-\infty, -3) \cup (-3, 4)\cup (4, \infty)$


\item   $(-\infty, 0) \cup (0, 2)\cup (2, \infty)$

\setcounter{HW}{\value{enumi}}
\end{enumerate}
\end{multicols}


\begin{multicols}{2}
\begin{enumerate}
\setcounter{enumi}{\value{HW}}

\item  $(-\infty, -2) \cup (-2, 2)\cup (2, \infty)$

\item  $(-\infty, -4) \cup (-4, 0) \cup (0, 4) \cup (4, \infty)$

\setcounter{HW}{\value{enumi}}
\end{enumerate}
\end{multicols}

\begin{multicols}{2}
\begin{enumerate}
\setcounter{enumi}{\value{HW}}

\item $(-\infty, -1] \cup [1, \infty)$

\item $(-\infty, \infty)$


\setcounter{HW}{\value{enumi}}
\end{enumerate}
\end{multicols}


\begin{multicols}{2}
\begin{enumerate}
\setcounter{enumi}{\value{HW}}

\item $(-\infty, -3] \cup (0, \infty)$

\item $(-\infty, 5] \cup \{6\}$

\setcounter{HW}{\value{enumi}}
\end{enumerate}
\end{multicols}

\begin{multicols}{2}
\begin{enumerate}
\setcounter{enumi}{\value{HW}}


\item $\{-1\} \cup \{1\} \cup (2, \infty)$

\item $(-3,3) \cup \{4\}$

\setcounter{HW}{\value{enumi}}
\end{enumerate}
\end{multicols}


\begin{enumerate} 
\setcounter{enumi}{\value{HW}}
\item The required points $\;A(-3, -7)$, $\;B(1.3, -2)$, $\;C(\pi, \sqrt{10})$, $\;D(0, 8)$, $\;E(-5.5, 0)$, $\;F(-8, 4)$, $\;G(9.2, -7.8)$, and $H(7, 5)$ are plotted in the Cartesian Coordinate Plane below. 

\begin{center}

\begin{mfpic}[20]{-10}{10}{-10}{10}
\axes
\tlabel[cc](10,-0.5){\scriptsize $x$}
\tlabel[cc](0.5,10){\scriptsize $y$}
\xmarks{-9,-8,-7,-6,-5,-4,-3,-2,-1,1,2,3,4,5,6,7,8,9}
\ymarks{-9,-8,-7,-6,-5,-4,-3,-2,-1,1,2,3,4,5,6,7,8,9}
\gfill \circle{(-3, -7),0.1}
\tlabel[cc](-3, -7.75){$A(-3,-7)$}
\gfill \circle{(1.3,-2),0.1}
\tlabel[cc](1.5, -2.5){$B(1.3, -2)$}
\gfill \circle{(3.14159, 3.16228),0.1}
\tlabel[cc](3.14, 2.7){$C(\pi, \sqrt{10})$}
\gfill \circle{(0, 8),0.1}
\tlabel[cc](1.25, 8){$D(0, 8)$}
\gfill \circle{(-5.5,0),0.1}
\tlabel[cc](-5.5, 0.5){$E(-5.5,0)$}
\gfill \circle{(-8,4),0.1}
\tlabel[cc](-8, 3.5){$F(-8, 4)$}
\gfill \circle{(9.2,-7.8),0.1}
\tlabel[cc](9.2, -8.3){$G(9.2, -7.8)$}
\gfill \circle{(7 ,5),0.1}
\tlabel[cc](7, 5.5){$H(7, 5)$}
\tlpointsep{5pt}
\scriptsize
\axislabels {x}{{$-9 \hspace{7pt}$} -9, {$-8 \hspace{7pt}$} -8, {$-7 \hspace{7pt}$} -7, {$-6 \hspace{7pt}$} -6, {$-5 \hspace{7pt}$} -5, {$-4 \hspace{7pt}$} -4, {$-3 \hspace{7pt}$} -3, {$-2 \hspace{7pt}$} -2, {$-1 \hspace{7pt}$} -1, {$1$} 1, {$2$} 2, {$3$} 3, {$4$} 4, {$5$} 5, {$6$} 6, {$7$} 7, {$8$} 8, {$9$} 9}
\axislabels {y}{{$-9$} -9, {$-8$} -8, {$-7$} -7, {$-6$} -6, {$-5$} -5, {$-4$} -4, {$-3$} -3, {$-2$} -2, {$-1$} -1, {$1$} 1, {$2$} 2, {$3$} 3, {$4$} 4, {$5$} 5, {$6$} 6, {$7$} 7, {$8$} 8, {$9$} 9}
\normalsize
\end{mfpic}

\end{center}

\pagebreak

\small %In order to fit everything on one page, we made it smaller.

\item \begin{multicols}{2}

\begin{enumerate}

\item The point $A(-3, -7)$ is 

\begin{itemize}

\item in Quadrant III
\item symmetric about $x$-axis with $(-3, 7)$
\item symmetric about $y$-axis with $(3, -7)$
\item symmetric about origin with $(3, 7)$

\end{itemize}

\item The point $B(1.3, -2)$ is 

\begin{itemize}

\item in Quadrant IV
\item symmetric about $x$-axis with $(1.3, 2)$
\item symmetric about $y$-axis with $(-1.3, -2)$
\item symmetric about origin with $(-1.3, 2)$

\end{itemize}

\setcounter{HWindent}{\value{enumii}}
\end{enumerate}
\end{multicols}

\begin{multicols}{2}
\begin{enumerate}
\setcounter{enumii}{\value{HWindent}}

\item The point $C(\pi, \sqrt{10})$ is 

\begin{itemize}

\item in Quadrant I
\item symmetric about $x$-axis with {\small $(\pi, -\sqrt{10})$}
\item symmetric about $y$-axis with {\small $(-\pi, \sqrt{10})$}
\item symmetric about origin with {\scriptsize $(-\pi, -\sqrt{10})$}

\end{itemize}

\item The point $D(0, 8)$ is 

\begin{itemize}

\item on the positive $y$-axis
\item symmetric about $x$-axis with $(0, -8)$
\item symmetric about $y$-axis with $(0, 8)$
\item symmetric about origin with $(0, -8)$

\end{itemize}


\setcounter{HWindent}{\value{enumii}}
\end{enumerate}
\end{multicols}

\begin{multicols}{2}
\begin{enumerate}
\setcounter{enumii}{\value{HWindent}}

\item The point $E(-5.5, 0)$ is 

\begin{itemize}

\item on the negative $x$-axis
\item symmetric about $x$-axis with $(-5.5, 0)$
\item symmetric about $y$-axis with $(5.5, 0)$
\item symmetric about origin with $(5.5, 0)$

\end{itemize}

\item The point $F(-8, 4)$ is 

\begin{itemize}

\item in Quadrant II
\item symmetric about $x$-axis with $(-8, -4)$
\item symmetric about $y$-axis with $(8, 4)$
\item symmetric about origin with $(8, -4)$

\end{itemize}

\setcounter{HWindent}{\value{enumii}}
\end{enumerate}
\end{multicols}

\begin{multicols}{2}
\begin{enumerate}
\setcounter{enumii}{\value{HWindent}}

\item The point $G(9.2, -7.8)$ is 

\begin{itemize}

\item in Quadrant IV
\item symmetric about $x$-axis with $(9.2, 7.8)$
\item symmetric about $y$-axis with {\scriptsize $(-9.2, -7.8)$}
\item symmetric about origin with $(-9.2, 7.8)$

\end{itemize}

\item The point $H(7, 5)$ is 

\begin{itemize}

\item in Quadrant I
\item symmetric about $x$-axis with $(7, -5)$
\item symmetric about $y$-axis with $(-7, 5)$
\item symmetric about origin with $(-7, -5)$

\end{itemize}

\end{enumerate}
\end{multicols}
\setcounter{HW}{\value{enumi}}
\end{enumerate}


\begin{multicols}{2}
\begin{enumerate}
\setcounter{enumi}{\value{HW}}

\item $d = 5$, $M = \left(-1, \frac{7}{2} \right)$
\item $d = 4 \sqrt{10}$, $M = \left(1, -4 \right)$

\setcounter{HW}{\value{enumi}}
\end{enumerate}
\end{multicols}

\begin{multicols}{2}
\begin{enumerate}
\setcounter{enumi}{\value{HW}}

\item $d = \sqrt{26}$, $M = \left(1, \frac{3}{2} \right)$
\item $d= \frac{\sqrt{37}}{2}$, $M = \left(\frac{5}{6}, \frac{7}{4} \right)$

\setcounter{HW}{\value{enumi}}
\end{enumerate}
\end{multicols}

\begin{multicols}{2}
\begin{enumerate}
\setcounter{enumi}{\value{HW}}

\item  $d = \sqrt{74}$, $M = \left(\frac{13}{10}, -\frac{13}{10} \right)$ \vphantom{$\left( \frac{\sqrt{3}}{2} \right)$}
\item $d= 3\sqrt{5}$, $M = \left(-\frac{\sqrt{2}}{2}, -\frac{\sqrt{3}}{2} \right)$

\setcounter{HW}{\value{enumi}}
\end{enumerate}
\end{multicols}

\begin{multicols}{2}
\begin{enumerate}
\setcounter{enumi}{\value{HW}}

\item  $d = \sqrt{83}$, $M = \left(4 \sqrt{5}, \frac{5 \sqrt{3}}{2} \right)$
\item $d = \sqrt{x^2 + y^2}$, $M = \left( \frac{x}{2}, \frac{y}{2}\right)$ \vphantom{$\left( \frac{\sqrt{3}}{2} \right)$}

\setcounter{HW}{\value{enumi}}
\end{enumerate}
\end{multicols}


\begin{multicols}{2}
\begin{enumerate}
\setcounter{enumi}{\value{HW}}

\item  $(3 + \sqrt{7}, -1)$, $(3-\sqrt{7}, -1)$
\item $(0,3)$

\setcounter{HW}{\value{enumi}}
\end{enumerate}
\end{multicols}


\begin{multicols}{2}
\begin{enumerate}
\setcounter{enumi}{\value{HW}}

\item $(-1+\sqrt{3},0)$, $(-1-\sqrt{3},0)$ \vphantom{$\left( \frac{\sqrt{3}}{2} \right)$}
\item $\left(\frac{\sqrt{2}}{2},-\frac{\sqrt{2}}{2} \right)$, $\left(-\frac{\sqrt{2}}{2},\frac{\sqrt{2}}{2}\right)$

\setcounter{HW}{\value{enumi}}
\end{enumerate}
\end{multicols}

\begin{enumerate}
\setcounter{enumi}{\value{HW}}

\item $(-3, -4)$, $5$ miles, $(4, -4)$


\addtocounter{enumi}{2}

\item  \begin{enumerate}  

\item  The distance from $A$ to $B$ is $|AB| = \sqrt{13}$, the distance from $A$ to $C$ is $|AC| = \sqrt{52}$, and the distance from $B$ to $C$ is $|BC| = \sqrt{65}$.  Since $\left(\sqrt{13}\right)^2 + \left( \sqrt{52} \right)^2 = \left( \sqrt{65} \right)^2$, we are guaranteed by the \href{http://en.wikipedia.org/wiki/Pythagorean_theorem#Converse}{\underline{converse of the Pythagorean Theorem}} that the triangle is a right triangle.
\item Show that $|AC|^{2} + |BC|^{2} = |AB|^{2}$

\end{enumerate}

\end{enumerate}

\normalsize

\closegraphsfile